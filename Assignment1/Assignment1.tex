\documentclass[journal,12pt,twocolumn]{IEEEtran}

\usepackage{setspace}
\usepackage{gensymb}
\singlespacing
\usepackage[cmex10]{amsmath}

\usepackage{amsthm}

\usepackage{mathrsfs}
\usepackage{txfonts}
\usepackage{stfloats}
\usepackage{bm}
\usepackage{cite}
\usepackage{cases}
\usepackage{subfig}

\usepackage{longtable}
\usepackage{multirow}

\usepackage{enumitem}
\usepackage{mathtools}
\usepackage{steinmetz}
\usepackage{tikz}
\usepackage{circuitikz}
\usepackage{verbatim}
\usepackage{tfrupee}
\usepackage[breaklinks=true]{hyperref}
\usepackage{graphicx}
\usepackage{tkz-euclide}

\usetikzlibrary{calc,math}
\usepackage{listings}
    \usepackage{color}                                            %%
    \usepackage{array}                                            %%
    \usepackage{longtable}                                        %%
    \usepackage{calc}                                             %%
    \usepackage{multirow}                                         %%
    \usepackage{hhline}                                           %%
    \usepackage{ifthen}                                           %%
    \usepackage{lscape}     
\usepackage{multicol}
\usepackage{chngcntr}

\DeclareMathOperator*{\Res}{Res}

\renewcommand\thesection{\arabic{section}}
\renewcommand\thesubsection{\thesection.\arabic{subsection}}
\renewcommand\thesubsubsection{\thesubsection.\arabic{subsubsection}}

\renewcommand\thesectiondis{\arabic{section}}
\renewcommand\thesubsectiondis{\thesectiondis.\arabic{subsection}}
\renewcommand\thesubsubsectiondis{\thesubsectiondis.\arabic{subsubsection}}


\hyphenation{op-tical net-works semi-conduc-tor}
\def\inputGnumericTable{}                                 %%

\lstset{
%language=C,
frame=single, 
breaklines=true,
columns=fullflexible
}

\begin{document}
\title{Assignment 1}
\author{Muttareddy Sahil Chandra - CS20BTECH11033}
\maketitle
\newpage
\bigskip
\renewcommand{\thefigure}{\theenumi}
\renewcommand{\thetable}{\theenumi}
Download all python codes from 
\begin{lstlisting}
https://github.com/SAHIL150602/AI1103/blob/main/Assignment1/codes/Assignment.py
\end{lstlisting}
%
and latex-tikz codes from 
%
\begin{lstlisting}
https://github.com/SAHIL150602/AI1103/blob/main/Assignment1/Assignment1.tex
\end{lstlisting}
\section{Problem}
How many times a man must toss a coin such that the probability of getting atleast one head is greater than $90\%$
\section{Solution} 
The probability of getting heads when an unbiased coin is tossed 1 time  is {\large$\frac{1}{2}$}  and  vice versa.
  
  By Binomial distribution the probability of getting $k$ heads is
  
      \begin{align}
        \Pr(X = k) & =\;^nC_k\left(\frac{1}{2}\right)^k\left(\frac{1}{2}\right)^{n-k}\\  
        & =\frac{^nC_k}{2^n}\\  
      \end{align}
  
     
  So the probability of getting atleast one head is $P(X>0) $.
  
  By binomial distribution we also know that
  \begin{align}
      \sum_{i=0}^{n}P(X=i) &= 1
  \end{align}
  
  So    
  \begin{align}
      \Pr(X\ge 1) + \Pr(X=0) &= 1\\
      \Pr(X\ge 1) &= 1 - \Pr(X=0)\\
      &= 1 - \frac{^nC_0}{2^n}\\
      \Pr(X\ge 1) &= 1 - \frac{1}{2^n}
  \end{align}
  the condition given was probability of getting at least one head must be greater than  $90\%$
 \begin{align}
      &\implies \Pr(X\ge 1)>0.9\\
      &\implies 1-\frac{1}{2^n}>0.9\\
      &\implies\frac{1}{2^n}<0.1\\
      &\implies 2^n>10
  \end{align}
  Hence $n =4$\\
$\implies$ The minimum number of times the coin has to be tossed so that the probability  of getting atleast one head is greater than 0.9 is 4 times.    
\end{document}


